\documentclass{article}
\usepackage[english]{babel}
\usepackage{csquotes}
\usepackage{biblatex}
%\addbibresource{mybib/BibEntries.bib}
\addbibresource{local.bib}


\usepackage{amsmath}
\usepackage{amsthm}
\usepackage{latexsym}
\usepackage{amssymb}

%% Mathematical Domains
\newcommand{\ufd}{\mathbb{D}}
\newcommand{\ufdfrac}{\mathbb{F}}
\newcommand{\intring}{\mathbb{Z}}
\newcommand{\ratring}{\mathbb{Q}}
\newcommand{\realring}{\mathbb{R}}
\newcommand{\compring}{\mathbb{C}}

%% Systems
\newcommand{\qepcad}{\textsc{Qepcad}}
\newcommand{\qepcadb}{\textsc{Qepcad b}}
\newcommand{\saclib}{\textsc{Saclib}}
\newcommand{\slfq}{\textsc{SLFQ}}
\newcommand{\tarski}{\textsc{Tarski}}


\title{Projection for Cylindrical Algebraic Decomposition\\
  \medskip An Annotated Bibliography}
\author{Christopher W.~Brown, United States Naval Academy}

\begin{document}
\maketitle

%%%%%%%%%%%%%%%%%%%%%%%%%%%%%%%%%%%%%%%%%%%%%%%%%%%%%%%%%%%%%%%
%% Introduction
%%%%%%%%%%%%%%%%%%%%%%%%%%%%%%%%%%%%%%%%%%%%%%%%%%%%%%%%%%%%%%%
\section{Introduction}
This document is an annotated bibliography of papers dealing with
projection in Cylindrical Algebraic Decomposition (CAD).  Its intended
audience is people already familiar with the concepts and terminology
of CAD, and the discussions appearing here are written with
that audience in mind.  Here are some important points to note when
comparing and contrasting different projections:

\begin{enumerate}
\item There are (in my view) three families of projections:
  Collins-style, McCallum-style and Lazard-style projections.  My
  organization of the references is organized this way.
  One could possibly add a fourth family based on triangular
  decomposition. 
\item A projection operator cannot be wholly separated from the process of
  constructing cell representations from the projection.  For
  example, using ``Lazard projection'' requires ``Lazard lifting'',
  while using the ``model-based projection'' requires a cell
  construction procedure that is ``model-based'', i.e. that starts with a
  sample point and then computes polynomials from it, rather than
  first computing polynomials and then constructing sample points from
  them. So, although the language of the paper may refer to a
  particular projection, implicitly it is also referring to the
  associated cell-construction process.
\item One issue that comes up when discussing different projections is
  \emph{completeness}, which is generally taken to mean that,
  starting with a formula of the appropriate kind, the projection
  is always able to produce a truth-invariant CAD for the formula.
  In the model-based paradigm, ``complete'' typically
  means that there is a complete non-model-based projection such that 
  the cell the model-based algorithm produces around the model point is a superset of the
  cell from the CAD the non-model-based projection would produce from
  the same formula that contains the model point.
\item An issue that is, in my view, often overlooked is the question
  of constructing formulas from CADs or of constructing formulas for 
  individual cells.  This issue depends on the target language for
  these formulas as well as the projection operator used and
  properties of the specific CAD from which a formula is to be
  constructed.  Typically, these formulas are either required to
  be \emph{Tarski formulas} or are allowed to also contain
  \emph{indexed-root expressions}.  In this work, we will say that
  a CAD-with-truth-values is \emph{projection definable for X}, where ``X'' is a type
  of formula, e.g. ``Tarski formula'', if a formula of type ``X'' defining the union
  of the true cells can be constructed using only projection factors.
\end{enumerate}

The remainder of this document consists of sections for the three
families of projection operators: Collins, McCallum and Lazard.
Cross-cutting issues, e.g., model-based versions of projections,
appear based on which family of projection operator is used.


%%%%%%%%%%%%%%%%%%%%%%%%%%%%%%%%%%%%%%%%%%%%%%%%%%%%%%%%%%%%%%%
%% Collins-style
%%%%%%%%%%%%%%%%%%%%%%%%%%%%%%%%%%%%%%%%%%%%%%%%%%%%%%%%%%%%%%%
\section{Collins-style projection}
George Collins introduced CAD in the early 1970s, and described it in
a series of presentations and papers:
%
\begin{description}
\item[\autocite{Collins:73b} :]  \fullcite{Collins:73b}
\item[\autocite{Collins:74b} :]  \fullcite{Collins:74b}
\item[\autocite{Collins:75} :]  \fullcite{Collins:75}
\end{description}
%
His projection operator was based on \emph{sign invariance}, and was
coupled with a lifting procedure based on partial evaluation of
polynomials (at rational and algebraic points) and univariate
polynomial real root isolation.  The projection adds coefficients,
discriminants, resultants, subresultant coefficients and
sub-discriminant coefficients\footnote{Often overlooked is the fact that,
when constructing a CAD for quantifier elimination,
he included an even larger ``augmented projection'' to be used for
projections in free-variable space to ensure that a solution formula
consisting solely of polynomial sign constraints could be
constructed.}
This projection/lifting scheme is \emph{complete} in
the sense that for any input Tarski formula, it produces a CAD with
cells in which the polynomials from the formula have constant sign,
and thus the formula has constant truth value.  When the ``augmented
projection'' is used, each cell has a defining Tarski formula that can be
constructed from projection factors and their derivatives.  However, I
know of no implementation that uses the augmented projection.  Without
the augmented projection, the resulting CAD is not, in general,
projection definable for Tarski formulas, though it is projection
definable for Tarski formulas + indexed root expressions.


An improved version of the Collins projection was described by Hong:
%
\begin{description}
\item[\autocite{Hong:90a} :]  \fullcite{Hong:90a}
\end{description}
%
The Hong projection follows the basic principles of the Collins
projection, but avoids some coefficients, subdiscriminant coefficients
and subresultant coefficients by providing criteria that allow one to
determine that they are unnecessary.  This produces CADs with fewer
cells, and does so in less time.  The resulting projection/lifting
scheme is still complete.  Like the Collins projection without
augmented projection, the Hong projection produces CADs that are not,
in general,
projection definable for Tarski formulas, though they are projection
definable for Tarski formulas + indexed root expressions.

The first model-based projection was based on the Hong projection and
introduced in:
%
\begin{description}
\item[\autocite{JovanovicdeMoura:12} :]  \fullcite{JovanovicdeMoura:12}
\item[\autocite{Jovanovic:2013} :]  \fullcite{Jovanovic:2013}
\end{description}
%
Its authors dubbed it \emph{the model-based projection}, but because
we want to use that term to refer to a number of projection
operators that act similarly, we will refer to it here as the
\emph{Jovanovi\'c and de Moura Projection}.
Model-based projections disrupt the project-then-lift paradigm of
Collins' original CAD construction algorithm and the variants that had
followed it up to that point.  In model-based projection, we start
with a ``model point'', and the goal is to construct a (hopefully
large) CAD cell containing the model point and satisfying some other
properties.  In the Jovanovi\'c and de Moura Projection, those other
properties are the sign-invariance of an initial set of polynomials.
This projection uses the model point to determine that certain
polynomials that are included in the Hong projection can be left out.
For example, in projecting polynomial $p := x z^2 + (x - y) z + x y - 1$ with respect
to $z$ using model point (1,1,1), the Hong projection would include
$x$ and $x - y$ , the coefficient of the quadratic and
linear terms of $p$ as a polynomial in $z$.  Since leading coefficient
$x$ is non-zero at the model
point and $x$ is included in the projection, the 
Jovanovi\'c and de Moura Projection deduces that coefficient $x-y$ can
be left out of the projection without losing the desired properties of
the cell that will be constructed around the model point.  

%%%%%%%%%%%%%%%%%%%%%%%%%%%%%%%%%%%%%%%%%%%%%%%%%%%%%%%%%%%%%%%
%% McCallum-style
%%%%%%%%%%%%%%%%%%%%%%%%%%%%%%%%%%%%%%%%%%%%%%%%%%%%%%%%%%%%%%%
\section{McCallum-style projection}
The McCallum projection is described in:
%
\begin{description}
\item[\autocite{McCallum:84} :]  \fullcite{McCallum:84}
\item[\autocite{McCallum:88} :]  \fullcite{McCallum:88}
\item[\autocite{McCallum:97} :]  \fullcite{McCallum:97}
\end{description}
%
The McCallum projection is based on \emph{order invariance} and is
coupled with a lifting procedure that follows Collins' original
lifting procedure with two differences:
(1) when a projection factor derived from a projection step (as
opposed to appearing solely as an input polynomial) is nullified over a cell
of positive dimension, the lifting procedure exits with ``fail'', and
(2) when a projection factor $p$ derived from a projection step is
nullified over 
a cell $c$ of dimension zero, instead of including the roots of $p(c,x)$ as
sections in the decomposition of $c\times\realring$, a set called
\emph{the delineating polynomials} are constructed, and their roots
above $c$ are used as sections.  The set of delineating polynomials
consists of all partial derivatives of $p$ of some order $k$, where $k$ is
determined by $p$ and $c$.

The McCallum projection includes coefficients, discriminants and
resultants, and so is a subset of the Collins projection and the Hong
projection.  It has the advantage of producing a smaller
projection.  However, it is not complete.  As described above, it may
fail during the lifting process.  (Note that if it returns without
failure the decomposition is guaranteed to be correct.)

The question of solution formula construction is interesting.  If the
McCallum projection does not fail, and it does not need to use
delineating polynomials, the resulting CAD will be projection
definable for Tarski formulas + indexed root expressions, just as with
Collins-style projection.  However, if the CAD has a cell that is a
section of a delineating polynomial and not a section of any
projection factor, then that cell (and the sectors above and below it)
cannot be described by a formula including only projection factors.
However, it can be described by a formula using indexed roots of
projection factors and an indexed-root expression for one of the
delineating polynomials of which it is a section.  Moreover, there is
a procedure for refining the CAD until it is projection definable
by Tarski formulas.  This is described in \autocite{Brown:99a}.

Nullification of a projection factor is detected and dealt with during
lifting.  The technical report:
%
\begin{description}
\item[\autocite{Brown:01} :]  \fullcite{Brown:01}
\end{description}
%
... describes techniques for determining, in some instances, that
nullification does not invalidate projection/lifting, and it
provides an alternative to McCallum's delineating polynomials for
handling nullifications of projection factors over
positive-dimensional regions.  

% Brown-McCallum
An improved version of McCallum's projection, which we will refer to
here as the Brown-McCallum projection, was described in
%
\begin{description}
\item[\autocite{Brown:00a} :]  \fullcite{Brown:00a}
\item[\autocite{Brown:00b} :]  \fullcite{Brown:00b}
\end{description}
%
... which shows that when projecting polynomial $p(x_1,\ldots,x_n)$, the leading
coefficient in $x_n$ is the only coefficient required for McCallum's
projection, provided that (1) some analysis of the system of all
coefficients in $x_n$ of $p$ is done prior to projection, and (2) if
that system has only single-point solutions, those points are added
as ``projection points''.  Of course if the system has a
positive-dimensional set of solutions then McCallum's original
projection returns ``fail'', as does the improved projection.
The Brown-McCallum projection is a subset of McCallum's
projection, but it requires the up-front analysis of systems of
coefficients. This can be done efficiently relative to the cost of
including extra coefficients in projection, but adds to the complexity
of implementations. Given a CAD resulting from the
Brown-McCallum projection, a defining formula can be constructed from
indexed-root expressions of projection factors plus indexed-root
expressions for the coordinates of the ``projection points''. 

McCallum's projection has been improved, specialized and adapted in a
number of ways.  In the following we

\subsection{Equational constraints and McCallum's projection}
Improving the performance of CAD construction by explotining
``equational constraints'' was suggested by Collins in
%
\begin{description}
\item[\autocite{Collins:94} :]  \fullcite{Collins:94}
\end{description}
%
... though only as a suggested avenue of inquiry. What constitutes an
``equational constraint'' is technical enough that I'll leave it to
the papers --- except to say that, for polynomial $p$,  if the input formula has the
property that $p(\alpha) = 0$ for any point $\alpha$ satisfying the
formula, then $p$ is an equational constraint for that formula.
McCallum developed this idea
and proved basic results about how to exploit equational constraints
during projection and lifting.
%
\begin{description}
\item[\autocite{McCallum:99} :]  \fullcite{McCallum:99}
\item[\autocite{McCallum:01} :]  \fullcite{McCallum:01}
\end{description}
%
The first of these papers shows how an $n$-level equational constraint
polymomial can be used to reduce the projection at level $n$.
The second of these shows how, in the presence of multiple
$n$-level equational constraints, new equational constraints at lower
levels are created by the projection process.  These are called
propagated constraints, and while top-level equational constraint
polynomials typically only need to be sign invariant within cells,
propagated constraints need to be order invariant.  So propagated
constraints need to be handled differently than top-level constraints. 

Further development of McCallum's  approach, including results on
how the complexity of CAD construction is improved by exploiting
equational constraints, appears in:
%
\begin{description}
\item[\autocite{EnglandEtAl:2020} :]  \fullcite{EnglandEtAl:2020}
\end{description}
%

When there are multiple equational constraint polynomials at the same
level, the original approach to equational constraints designates one
of them as \emph{the} constraint for projection purposes.  An
alternative is to try to formulate a projection operator that can
leverage all constraints at that level to reduce the projection.
The following two papers follow that approach for the case of two
equational constraints.

% Equational-constraints (including multi-equational constraints)
%
\begin{description}
\item[\autocite{BrownMcCallum:05} :]  \fullcite{BrownMcCallum:05}
\item[\autocite{BrownMcCallum:09} :]  \fullcite{BrownMcCallum:09}
\end{description}
%


\subsection{Model-based variants of McCallum's projection}
The model-based projection introduced in
Jovanovi\'c and de Moura's NLSAT alrogithm was adaptation of
The Collins-Hong projection to make use of the fact that, ultimately,
only a single cell was being constructed and, since the sample point
for that cell is given as input, its coordinates can give information
about polynomials at or near that point.
This same idea has been explored in the context of McCallum's
projection as well.  Strzebonski proposed a model-based version of
McCallum's projection to be applied to the general problem of
constructing CADs in:
% Strzebonski model-based projection Strzebonski:2014
% One-cell Brown:2013 BrownKosta:2014
% Cylindrical coverings AbrahamEtAl:2021
%
\begin{description}
\item[\autocite{Strzebonski:2014} :]  \fullcite{Strzebonski:2014}
\end{description}
%
A model-based version of McCallum's projection applied specifically to
the problem of constructing a single cell around a model point was
given, first for open cells, and then for general cells in:
\begin{description}
\item[\autocite{Brown:2013} :]  \fullcite{Brown:2013}
\item[\autocite{BrownKosta:2014} :]  \fullcite{BrownKosta:2014}
\end{description}
%
In their introduction of Cylindrical Algebraic Coverings,
{\'A}brah{\'a}m et al. give a different model-based variant of the
McCallum projection for the problem of constructing a ``covering'' by
generalized intervals, rather than a single cell:
\begin{description}
\item[\autocite{AbrahamEtAl:2021} :]  \fullcite{AbrahamEtAl:2021}
\end{description}
%


\subsection{Pojective delineability}
% Projective Delineablility
Michel et al. propose a variant of McCallum's projection aimed at
providing, not delineability, but ``projective delineability''.
In the usual sense of delineability, we have a total ordering of cells
decomposing $S\times\realring$, where $S$ is some base cell.  In
projective delineability, we have 
a cyclic ordering of cells decomposing $S\times\mathbb{P}$, where
$\mathbb{P}$ is real projective space.
%
\begin{description}
\item[\autocite{MichelEtAl:2024} :]  \fullcite{MichelEtAl:2024}
\end{description}
%

\subsection{McCallum-style projection for ``open'' CAD}
% McCallum, Strzebonski, Open OneCell, Deducing open projections?
An ``Open CAD'' is a form of partial CAD in which only the full
dimensional cells (which are exactly the cells that are open sets)
matter, where ``matter'' is problem dependent.  For example, in
deciding the satisfiability of a formula in which all atoms are strict
inequalities ($<$, $\leq$, or $\neq$), only the full dimensional cells
``matter'' because there is a solution if and only if there is an open set
of solutions.  So we can answer the question without considering any
cell of less than full dimension. 
Projection (and lifting) becomes a lot easier when we restrict to only
cells of full dimension.   There have been several papers that have
looked at adapting McCallum's projection to this easier case.
%
\begin{description}
\item[\autocite{McCallum:93} :]  \fullcite{McCallum:93}
\item[\autocite{Strzebonski:94} :]  \fullcite{Strzebonski:94}
\item[\autocite{Brown14} :]  \fullcite{Brown14}
\item[\autocite{BarEtAl:2023} :]  \fullcite{BarEtAl:2023}
\end{description}
%
%%%%%%%%%%%%%%%%%%%%%%%%%%%%%%%%%%%%%%%%%%%%%%%%%%%%%%%%%%%%%%%
%% Lazard-style
%%%%%%%%%%%%%%%%%%%%%%%%%%%%%%%%%%%%%%%%%%%%%%%%%%%%%%%%%%%%%%%
\section{Lazard-style projection}
In the 1990s, Daniel Lazard proposed a projection that was a strict
improvement on McCallum's projection in two ways: 1) that the
projection set is a subset of McCallum's, and 2) that it is valid for
any input set of polynomials, even when nullification occurs.
% LAzard original Lazard:94
%
\begin{description}
\item[\autocite{Lazard:94} :]  \fullcite{Lazard:94}
\end{description}
%
However, the original article contained an invalid proof, and it
took about 25 years for a correct proof to be worked out (see below).
This projection is based not on \emph{sign invariance} or
\emph{order invariance}, but rather on the invariance of the
\emph{Lazard valuation} of projection factors in cells.
Instead of \emph{delineability} or \emph{analytic delineability},
there is \emph{Lazard delineability}.  It also introduced a new
lifting mechanism based not only on substitution and univariate real
root isolation, but on \emph{Lazard evaluation}.  With these concepts,
Lazard proposed a projection consisting of
resultants, discriminants and leading and trailing coefficients of
projection polynomials.  This is a subset of McCallum's original
projection, which included all coefficients (though not of the
Brown-McCallum projection).

The completeness of Lazard's projection/lifting makes it attractive.
It does, however, have the downside that CADs produced by Lazard's
method are not, in general, projection definable for Tarski formulas
or for Tarski formulas + indexed root expressions (or at least there
is no algorithm at present).  Instead, one needs to use
``Lazard-valuation-indexed root expressions'', which are non-standard.
Though, in fairness, for many problem instances in which this issue
occurs, McCallum-based methods would fail anyways.

The correctness of Lazard's projection was established in the
two-paper sequence:
%
\begin{description}
% Hong-McCalllum McCallumHong2016
\item[\autocite{McCallumHong2016} :]  \fullcite{McCallumHong2016}
% McCallum et al.  McCallumEtAl:2019
\item[\autocite{McCallumEtAl:2019} :]  \fullcite{McCallumEtAl:2019}
\end{description}
%
Soon thereafter, a Lazard analogue to the Brown-McCallum projection
was given, i.e. testable conditions under which trailing
coefficients do not need to be included in Lazard's projection.
%
\begin{description}
% McCallum-Brown BrownMcCallum:2020
\item[\autocite{BrownMcCallum:2020} :]  \fullcite{BrownMcCallum:2020}
\end{description}
%

Given the success of specialized versions of McCallum's projection
that take advantage of ``equational constraints'', one might hope that
something similar could be developed for Lazard's projection. The
following two papers make progress in that direction.
%
% Equational constrints and Lazard NairDavenportSankaran:2020
\begin{description}
\item[\autocite{NairDavenportSankaran:2020} :]  \fullcite{NairDavenportSankaran:2020}
\item[\autocite{DavenportEtAl:2023} :]  \fullcite{DavenportEtAl:2023}
\end{description}
%


\printbibliography
\end{document}

